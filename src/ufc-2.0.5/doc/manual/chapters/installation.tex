\chapter{Installation}
\label{app:installation}
\index{installation}

The \ufc{} package consists of two parts, the main part being a single
header file called \texttt{ufc.h}. In addition, a set of Python
utilities to simplify the generation of \ufc{} code is provided.

Questions, bug reports and patches concerning the installation should
be directed to the \ufc{} mailing list at the address
\begin{code}
fenics@lists.launchpad.net
\end{code}

\section{Installing \ufc{}}

To install UFC, simply run
\begin{code}
scons
sudo scons install
\end{code}

This installs the header file ufc.h and a small set of Python
utilities (templates) for generating UFC code. Files will be installed
under the default prefix.

The installation prefix may be optionally specified, for example

\begin{code}
scons install prefix=~/local
\end{code}

Alternatively, just copy the single header file \texttt{src/ufc/ufc.h}
into a suitable include directory.

If you do not want to build and install the python extenstion module of \ufc{},
needed by, e.g., PyDOLFIN, you can write

\begin{code}
sudo enablePyUFC=No
sudo cons install
\end{code}

Help with available options and default arguments can be viewed by
\begin{code}
scons -h
\end{code}
